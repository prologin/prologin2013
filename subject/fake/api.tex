

%
% This file was generated using gen/make_tex.rtex
% Do not modify unless you are absolutely sure of what you are doing
%

\begin{minipage}{\linewidth}
\functitle{qi\_candidat}

\begin{lst-c++}
int qi_candidat(string nom, string prenom)
\end{lst-c++}

\noindent
\begin{tabular}[t]{@{\extracolsep{0pt}}>{\bfseries}lp{10cm}}
Description~: & Retourne le QI du candidat s'il est inférieur ou égal à 42 et IMPOSSIBLE sinon. \\


Parametres~: &
\begin{tabular}[t]{@{\extracolsep{0pt}}ll}
    
    
      
        \textsl{nom}~: & Nom du candidat \\
      
    
      
        \textsl{prenom}~: & Prénom du candidat \\
      
    
  \end{tabular} \\






\end{tabular} \\[0.3cm]
\end{minipage}



\noindent \begin{tabular}{lp{15cm}}
\textbf{Constante:} & FIN\_PARTIE \\
\textbf{Valeur:} & 500 \\
\textbf{Description:} & Nombre de tours par partie \\
\end{tabular} 
\vspace{0.2cm} \\



\noindent \begin{tabular}{lp{15cm}}
\textbf{Constante:} & MAX\_JOUEURS \\
\textbf{Valeur:} & 2 \\
\textbf{Description:} & Nombre maximum de joueurs dans la partie \\
\end{tabular} 
\vspace{0.2cm} \\



\noindent \begin{tabular}{lp{15cm}}
\textbf{Constante:} & LIMITE\_VAISSEAU \\
\textbf{Valeur:} & 5000 \\
\textbf{Description:} & Limite du nombre de vaisseaux pour chaque joueur \\
\end{tabular} 
\vspace{0.2cm} \\





\functitle{erreur} \\
\noindent
\begin{tabular}[t]{@{\extracolsep{0pt}}>{\bfseries}lp{10cm}}
Description~: & Erreurs possibles \\
Valeurs~: &
\small
\begin{tabular}[t]{@{\extracolsep{0pt}}lp{7cm}}
    
        \textsl{OK}~: & L'action a été exécutée avec succès \\
    
        \textsl{PLANETE\_INVALIDE}~: & La position spécifiée n'est pas une planète \\
    
        \textsl{POSITION\_INVALIDE}~: & La position spécifiée n'est pas valide \\
    
        \textsl{PLANETE\_ENNEMIE}~: & La planète ne vous appartient pas \\
    
        \textsl{MANQUE\_VAISSEAU}~: & Il n'y a pas assez de vaisseaux pour un déplacement \\
    
        \textsl{LIMITE\_ATTEINTE}~: & La limite de vaisseaux est atteinte \\
    
\end{tabular} \\
\end{tabular}





\functitle{position}

\begin{lst-c++}
struct position {
    double x;
    double y;
};
\end{lst-c++}

\noindent
\begin{tabular}[t]{@{\extracolsep{0pt}}>{\bfseries}lp{10cm}}
Description~: & Représente la position sur la carte \\
Champs~: &
\small
\begin{tabular}[t]{@{\extracolsep{0pt}}lp{7cm}}
    
        \textsl{x}~: & Coordonnée en X \\
    
        \textsl{y}~: & Coordonnée en Y \\
    
\end{tabular} \\
\end{tabular}



\functitle{planete}

\begin{lst-c++}
struct planete {
    int id;
    position pos;
    int joueur;
    int production;
};
\end{lst-c++}

\noindent
\begin{tabular}[t]{@{\extracolsep{0pt}}>{\bfseries}lp{10cm}}
Description~: & Représente une planète \\
Champs~: &
\small
\begin{tabular}[t]{@{\extracolsep{0pt}}lp{7cm}}
    
        \textsl{id}~: & Identifiant unique de la planète \\
    
        \textsl{pos}~: & Position \\
    
        \textsl{joueur}~: & Joueur \\
    
        \textsl{production}~: & Nombre de vaisseaux construits par tour \\
    
\end{tabular} \\
\end{tabular}


\begin{minipage}{\linewidth}
\functitle{info\_planete\_joueur}

\begin{lst-c++}
int info_planete_joueur(int id)
\end{lst-c++}

\noindent
\begin{tabular}[t]{@{\extracolsep{0pt}}>{\bfseries}lp{10cm}}
Description~: & Retourne le joueur qui possède la planète d'identifiant ``id``. Retourne -1 si la planète est libre ou si l'identifiant indiquée n'est pas une planète \\


Parametres~: &
\begin{tabular}[t]{@{\extracolsep{0pt}}ll}
    
    
      
        \textsl{id}~: & Identifiant de la planète \\
      
    
  \end{tabular} \\






\end{tabular} \\[0.3cm]
\end{minipage}


\begin{minipage}{\linewidth}
\functitle{info\_planete\_taille}

\begin{lst-c++}
int info_planete_taille(int id)
\end{lst-c++}

\noindent
\begin{tabular}[t]{@{\extracolsep{0pt}}>{\bfseries}lp{10cm}}
Description~: & Retourne la taille de la planète ayant pour identifiant ``id``. Retourne -1 si l'identifiant indiqué n'est pas une planète \\


Parametres~: &
\begin{tabular}[t]{@{\extracolsep{0pt}}ll}
    
    
      
        \textsl{id}~: & Identifiant de la planète \\
      
    
  \end{tabular} \\






\end{tabular} \\[0.3cm]
\end{minipage}


\begin{minipage}{\linewidth}
\functitle{info\_planete\_position}

\begin{lst-c++}
int info_planete_position(int id)
\end{lst-c++}

\noindent
\begin{tabular}[t]{@{\extracolsep{0pt}}>{\bfseries}lp{10cm}}
Description~: & Retourne la position de la planète ayant pour identifiant ``id`` \\


Parametres~: &
\begin{tabular}[t]{@{\extracolsep{0pt}}ll}
    
    
      
        \textsl{id}~: & Identifiant de la planète \\
      
    
  \end{tabular} \\






\end{tabular} \\[0.3cm]
\end{minipage}


\begin{minipage}{\linewidth}
\functitle{info\_planete\_vaisseaux}

\begin{lst-c++}
int info_planete_vaisseaux(int id)
\end{lst-c++}

\noindent
\begin{tabular}[t]{@{\extracolsep{0pt}}>{\bfseries}lp{10cm}}
Description~: & Retourne le nombre de vaisseaux sur la planète ayant pour identifiant ``id``. Retourne -1 si l'identifiant indiqué n'est pas une planète \\


Parametres~: &
\begin{tabular}[t]{@{\extracolsep{0pt}}ll}
    
    
      
        \textsl{id}~: & Identifiant de la planète \\
      
    
  \end{tabular} \\






\end{tabular} \\[0.3cm]
\end{minipage}


\begin{minipage}{\linewidth}
\functitle{nombre\_planetes}

\begin{lst-c++}
id array nombre_planetes()
\end{lst-c++}

\noindent
\begin{tabular}[t]{@{\extracolsep{0pt}}>{\bfseries}lp{10cm}}
Description~: & Retourne le nombre de planètes sur l'ensemble de la carte \\







\end{tabular} \\[0.3cm]
\end{minipage}


\begin{minipage}{\linewidth}
\functitle{liste\_planetes}

\begin{lst-c++}
id array liste_planetes()
\end{lst-c++}

\noindent
\begin{tabular}[t]{@{\extracolsep{0pt}}>{\bfseries}lp{10cm}}
Description~: & Retourne la liste des identifiants des planètes de la carte \\







\end{tabular} \\[0.3cm]
\end{minipage}


\begin{minipage}{\linewidth}
\functitle{mes\_planetes}

\begin{lst-c++}
id array mes_planetes()
\end{lst-c++}

\noindent
\begin{tabular}[t]{@{\extracolsep{0pt}}>{\bfseries}lp{10cm}}
Description~: & Retourne la liste des identifiants des planètes qui vous appartiennent \\







\end{tabular} \\[0.3cm]
\end{minipage}


\begin{minipage}{\linewidth}
\functitle{deplacer}

\begin{lst-c++}
erreur deplacer(position depart, position arrivee, int nb)
\end{lst-c++}

\noindent
\begin{tabular}[t]{@{\extracolsep{0pt}}>{\bfseries}lp{10cm}}
Description~: & Déplace un nombre ``nb`` de vaisseaux de la planète de départ à la planète d'arrivée \\


Parametres~: &
\begin{tabular}[t]{@{\extracolsep{0pt}}ll}
    
    
      
        \textsl{depart}~: & Départ \\
      
    
      
        \textsl{arrivee}~: & Arrivée \\
      
    
      
        \textsl{nb}~: & Nombre de vaisseau \\
      
    
  \end{tabular} \\






\end{tabular} \\[0.3cm]
\end{minipage}


\begin{minipage}{\linewidth}
\functitle{mon\_joueur}

\begin{lst-c++}
int mon_joueur()
\end{lst-c++}

\noindent
\begin{tabular}[t]{@{\extracolsep{0pt}}>{\bfseries}lp{10cm}}
Description~: & Retourne le numéro de votre joueur \\







\end{tabular} \\[0.3cm]
\end{minipage}


\begin{minipage}{\linewidth}
\functitle{adversaire}

\begin{lst-c++}
int adversaire()
\end{lst-c++}

\noindent
\begin{tabular}[t]{@{\extracolsep{0pt}}>{\bfseries}lp{10cm}}
Description~: & Retourne le numéro de votre adversaire \\







\end{tabular} \\[0.3cm]
\end{minipage}


\begin{minipage}{\linewidth}
\functitle{score}

\begin{lst-c++}
int score(int id_joueur)
\end{lst-c++}

\noindent
\begin{tabular}[t]{@{\extracolsep{0pt}}>{\bfseries}lp{10cm}}
Description~: & Retourne le score du joueur désigné par l'identifiant ``id`` \\


Parametres~: &
\begin{tabular}[t]{@{\extracolsep{0pt}}ll}
    
    
      
        \textsl{id\_joueur}~: & Identifiant du joueur \\
      
    
  \end{tabular} \\






\end{tabular} \\[0.3cm]
\end{minipage}


\begin{minipage}{\linewidth}
\functitle{tour\_actuel}

\begin{lst-c++}
int tour_actuel()
\end{lst-c++}

\noindent
\begin{tabular}[t]{@{\extracolsep{0pt}}>{\bfseries}lp{10cm}}
Description~: & Retourne le numéro du tour actuel \\







\end{tabular} \\[0.3cm]
\end{minipage}


\begin{minipage}{\linewidth}
\functitle{nombre\_vaisseaux}

\begin{lst-c++}
int nombre_vaisseaux(int id_joueur)
\end{lst-c++}

\noindent
\begin{tabular}[t]{@{\extracolsep{0pt}}>{\bfseries}lp{10cm}}
Description~: & Retourne le nombre de vaisseaux que possède le joueur désigné par l'identifiant ``id`` \\


Parametres~: &
\begin{tabular}[t]{@{\extracolsep{0pt}}ll}
    
    
      
        \textsl{id\_joueur}~: & Identifiant du joueur \\
      
    
  \end{tabular} \\






\end{tabular} \\[0.3cm]
\end{minipage}


