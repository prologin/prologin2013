%%%%%%%%%%%%%%%%%%%%%%%%%%%%%%%%%%%%%%%%
%% HOWTO                              %%
%%                                    %%
%% start the tree with \section       %%
%% do not use \chapter                %%
%% use \newpage between each \section %%
%%                                    %%
\if{0}
%% to insert pictures, use            %%
    \vspace{1cm} \begin{figure}[!h] \centering \includegraphics{.png} \caption{} \end{figure}
%%                                    %%
\fi
%%%%%%%%%%%%%%%%%%%%%%%%%%%%%%%%%%%%%%%%


\newcommand{\culdelampe}[0]{
        \begin{figure}[!h] \centering \includegraphics[height=1cm]{swirl.png} \end{figure}
}



\section{Introduction}

{\fontfamily{pzc}\selectfont \textit{Versailles, an de grâce 1692.}\\

\fontfamily{pzc}\selectfont \textit{La guerre contre l'Espagne fait rage, et tout ne se déroule pas comme prévu. Le souverain convoque son État-Major pour demander des comptes.}}\\

-- Sire, notre Marine Royale est submergée ! Lorsque nous attaquons
   des cibles militaires, les Espagnols en profitent pour détrousser
   nos marchands. Et bien entendu, défendre les marchands immobilise
   la flotte.

-- Et que proposez-vous donc, Amiral ?

-- La course, Sire.\\

Le souverain manqua de s'étouffer, et reprit vertement son sujet.

-- Amiral, la situation est grave, et ne se prête guère aux
   plaisanteries. Cessez incessamment vos sottises ! Gardes !\\

Alors que les gardes royaux approchaient et se saisissaient de
l'officier, le malheureux déglutit rapidement avant de reprendre la
parole avec hâte.

-- Sire, Sire ! Je n'oserais jamais me moquer en pareilles
   circonstances ! Il ne s'agit pas de la course de vitesse, bien que
   nos navires n'aient rien à envier à nos adversaires, mais de
   la\ldots{} piraterie organisée. Laissez-moi vous expliquer.

-- J'espère pour vous que cela aura l'heur de me plaire. Lâchez-le.\\

Transpirant sous son uniforme, l'Amiral réajusta son col d'une main,
et entreprit de présenter en quelques mots le principe des
corsaires.\\

-- Comme vous le voyez Sire, c'est comme la piraterie, sauf que rien
   ne sort de nos caisses, et l'État remporte malgré tout sa part du
   butin.

-- L'État, c'est moi.

-- Certes, Sire\ldots{} Vous remporterez votre part du butin.

-- Alors c'est décidé. Voyez avec le Ministre de la Marine pour faire
   la course.\\

Alors que Louis XIV quittait les lieux, l'Amiral et le Ministre de la
Marine abordaient les détails techniques de l'opération. Ils
convinrent de réquisitionner les meilleurs capitaines afin de leur
confier cette dangereuse mais ô combien lucrative mission.

\culdelampe{}

{\fontfamily{pzc}\selectfont \textit{Saint-Malo, an de grâce 1692.}}\\

-- Capitaine ! Un messager pour vous !

-- Faites-le monter, parbleu.

Le messager portait l'uniforme de la Marine Royale. Arrivant devant le
Capitaine, il commença à saluer conformément aux usages, mais fut
interrompu.

-- Au diable le protocole, parle donc !

-- Sa Majesté le Roy, Monsieur le Ministre de la Marine et Monsieur
   l'Amiral de la Marine Royale vous confient une mission de la plus
   haute importance.

-- Pas ici. Dans ma cabine.\\

Et le messager parla, expliqua, présenta.

La mission avait tout pour plaire. Pirater sans risque, avec
l'autorisation du Roy\ldots{} Alors certes, il faudra déclarer le
butin au Tribunal des prises de Saint-Malo pour que l'État y prélève
sa part\footnote{Sans oublier les traditionnelles CSG, CRDS, CSSS
!}\ldots{} Mais le Tribunal était loin d'être omniscient\ldots{}

Aussi, le Capitaine accepta. En retour, le messager fouilla dans sa
besace, et en sortit une liasse de documents, dont une lettre de
marque, et une bourse.\\
\newpage
{\fontfamily{pzc}\selectfont \begin{center}\Large{Lettre de Marque}\end{center}

Le Directoire Exécutif permet au détenteur de la présente de faire
armer et équiper en guerre les bâtiments qu'il lui sied, avec les
canons, boulets et quantité de poudre, plomb et autres munitions de
guerre et vivres qu'il jugera nécessaire pour les mettre en état de
courir sus tous les ennemis du Royaume et sur les pirates, forbans,
gens sans crédit, en quelque lieu qu'il pourra les rencontrer, de
prendre et amarrer prisonniers avec leurs navires, armes et autres
objets dont ils se seraient saisis ; à la charge du détenteur de se
conformer aux ordonnances et lois concernant la Marine, de faire
enregistrer la présente lettre au bureau de l'Inscription Maritime du
lieu de départ, d'y déposer un rôle signé et certifié du Capitaine,
contenant les noms et surnoms, âges, lieux de naissance et domicile
des gens de son équipage ; à la charge pour ledit Capitaine de faire
à son retour ou en cas de relâche, un rapport par devant
l'Administrateur de la Marine.\\

Le Directoire Exécutif donne également délégation au porteur afin de
prendre possession de terres et d'y installer une présence
française.\\

Le Directoire Exécutif invite toutes les puissances amies et alliées
du Royaume, et leurs agents, à donner audit Capitaine toute
assistance, passage et retraite en leurs ports, avec ses bâtiments et
les prises qu'il aura pu faire, offrant d'en user de même en pareille
circonstance.\\

Ordonne aux Commandants de vaisseaux du Royaume de France de laisser
passer le porteur de la présente avec ses bâtiments et ceux qu'il aura
pu prendre sur l'ennemi, et lui donner secours et assistance.\\

Ne pourra la présente servir que 36 heures seulement, à compter du 9
mai de l'an de grâce 2013.\\

En foi de quoi, le Directoire Exécutif a fait signer la présente
Lettre de Marque par le Ministre de la Marine.\\

Donné à Paris, le 9 mai de l'an de grâce 2013.\\

Pour le Ministre de la Marine,}

\newpage

Le reste des documents expliquait les règles de la course. Le
Capitaine y trouva notamment le rapport des explorateurs sur la zone
qui lui était confiée.

Que des îlots à explorer et conquérir. Et autant de caches possibles
pour ne pas déclarer le butin\footnote{Les politiciens n'ont rien
inventé.}, pensa-t-il.

Il devra construire et armer ses navires, sans autre aide que cette
petite bourse. Mais il entrevoyait des possibilités immenses, et une
fortune d'une taille équivalente. Il lui suffira de choisir
soigneusement ses îles vierges.\\

Alors qu'il tournait les talons afin de se rendre dans sa cabine, le
messager reprit la parole.

-- Un dernier mot, Capitaine. Votre lettre de marque n'est pas
   éternelle. À l'issue du délai, Sa Majesté le Roy récompensera
   généreusement le \st{contribuable} Capitaine avec le plus d'or.

\culdelampe{}

Le vaillant Capitaine arma son bâtiment, réunit son équipage, et fit
route vers sa Terre Promise.

Le voyage se fit sans encombre, néanmoins, alors qu'il arrivait en vue
de son île, la météo se dégrada soudain. Un vent puissant les poussait
vers un rocher. Impuissants, les marins ne purent éviter l'obstacle,
et durent rejoindre la terre ferme à la nage.\\

L'incident ne fit aucun blessé, mais tout le matériel gisait désormais
au fond de l'eau, irrécupérable. De tout ce qu'ils avaient apporté de
France, il ne restait plus que la royale bourse, et la Lettre de
Marque.\\


