

%
% This file was generated using gen/make_tex.rtex
% Do not modify unless you are absolutely sure of what you are doing
%



\noindent \begin{tabular}{lp{15cm}}
\textbf{Constante:} & TAILLE\_TERRAIN \\
\textbf{Valeur:} & 32 \\
\textbf{Description:} & Taille du terrain (longueur et largeur) \\
\end{tabular} 
\vspace{0.2cm} \\



\noindent \begin{tabular}{lp{15cm}}
\textbf{Constante:} & FIN\_PARTIE \\
\textbf{Valeur:} & 200 \\
\textbf{Description:} & Nombre de tours par partie \\
\end{tabular} 
\vspace{0.2cm} \\



\noindent \begin{tabular}{lp{15cm}}
\textbf{Constante:} & MAX\_JOUEURS \\
\textbf{Valeur:} & 2 \\
\textbf{Description:} & Nombre maximum de joueurs dans la partie \\
\end{tabular} 
\vspace{0.2cm} \\



\noindent \begin{tabular}{lp{15cm}}
\textbf{Constante:} & REVENU\_ILE \\
\textbf{Valeur:} & 10 \\
\textbf{Description:} & Revenu en or par île à chaque tour \\
\end{tabular} 
\vspace{0.2cm} \\



\noindent \begin{tabular}{lp{15cm}}
\textbf{Constante:} & REVENU\_VOLCAN \\
\textbf{Valeur:} & 20 \\
\textbf{Description:} & Revenu en or par volcan à chaque tour \\
\end{tabular} 
\vspace{0.2cm} \\



\noindent \begin{tabular}{lp{15cm}}
\textbf{Constante:} & CARAVELLE\_COUT \\
\textbf{Valeur:} & 3 \\
\textbf{Description:} & Coût de construction d'une Caravelle \\
\end{tabular} 
\vspace{0.2cm} \\



\noindent \begin{tabular}{lp{15cm}}
\textbf{Constante:} & GALION\_COUT \\
\textbf{Valeur:} & 1 \\
\textbf{Description:} & Coût de construction d'un Galion \\
\end{tabular} 
\vspace{0.2cm} \\



\noindent \begin{tabular}{lp{15cm}}
\textbf{Constante:} & CARAVELLE\_DEPLACEMENT \\
\textbf{Valeur:} & 4 \\
\textbf{Description:} & Déplacement de la Caravelle \\
\end{tabular} 
\vspace{0.2cm} \\



\noindent \begin{tabular}{lp{15cm}}
\textbf{Constante:} & GALION\_DEPLACEMENT \\
\textbf{Valeur:} & 6 \\
\textbf{Description:} & Déplacement du Galion \\
\end{tabular} 
\vspace{0.2cm} \\





\functitle{bateau\_type} \\
\noindent
\begin{tabular}[t]{@{\extracolsep{0pt}}>{\bfseries}lp{10cm}}
Description~: & Type de bateau \\
Valeurs~: &
\small
\begin{tabular}[t]{@{\extracolsep{0pt}}lp{7cm}}
    
        \textsl{BATEAU\_CARAVELLE}~: & Caravelle \\
    
        \textsl{BATEAU\_GALION}~: & Galion \\
    
        \textsl{BATEAU\_ERREUR}~: & Bateau inexistant \\
    
\end{tabular} \\
\end{tabular}



\functitle{terrain} \\
\noindent
\begin{tabular}[t]{@{\extracolsep{0pt}}>{\bfseries}lp{10cm}}
Description~: & Type de terrain \\
Valeurs~: &
\small
\begin{tabular}[t]{@{\extracolsep{0pt}}lp{7cm}}
    
        \textsl{TERRAIN\_ILE}~: & Île \\
    
        \textsl{TERRAIN\_VOLCAN}~: & Volcan \\
    
        \textsl{TERRAIN\_MER}~: & Mer \\
    
        \textsl{TERRAIN\_ERREUR}~: & Erreur, case impossible \\
    
\end{tabular} \\
\end{tabular}



\functitle{erreur} \\
\noindent
\begin{tabular}[t]{@{\extracolsep{0pt}}>{\bfseries}lp{10cm}}
Description~: & Erreurs possibles \\
Valeurs~: &
\small
\begin{tabular}[t]{@{\extracolsep{0pt}}lp{7cm}}
    
        \textsl{OK}~: & L'action a été exécutée avec succès \\
    
        \textsl{OR\_INSUFFISANT}~: & Vous ne possédez pas assez d'or pour cette action \\
    
        \textsl{ILE\_INVALIDE}~: & La position spécifiée n'est pas une île \\
    
        \textsl{POSITION\_INVALIDE}~: & La position spécifiée n'est pas valide \\
    
        \textsl{TROP\_LOIN}~: & La destination est trop éloignée \\
    
        \textsl{ILE\_COLONISEE}~: & L'île est déjà colonisée \\
    
        \textsl{ILE\_ENNEMIE}~: & L'île ne vous appartient pas \\
    
        \textsl{BATEAU\_ENNEMI}~: & L'île ne vous appartient pas \\
    
        \textsl{ID\_INVALIDE}~: & L'ID spécifiée n'est pas valide \\
    
        \textsl{NON\_DEPLACABLE}~: & Le bateau n'est pas déplaçable \\
    
        \textsl{AUCUNE\_CARAVELLE}~: & Il n'y a aucune caravelle susceptible de coloniser l'île \\
    
\end{tabular} \\
\end{tabular}





\functitle{position}

\begin{lst-c++}
struct position {
    int x;
    int y;
};
\end{lst-c++}

\noindent
\begin{tabular}[t]{@{\extracolsep{0pt}}>{\bfseries}lp{10cm}}
Description~: & Représente la position sur la carte \\
Champs~: &
\small
\begin{tabular}[t]{@{\extracolsep{0pt}}lp{7cm}}
    
        \textsl{x}~: & Coordonnée en X \\
    
        \textsl{y}~: & Coordonnée en Y \\
    
\end{tabular} \\
\end{tabular}



\functitle{bateau}

\begin{lst-c++}
struct bateau {
    int id;
    position pos;
    int joueur;
    bateau_type btype;
    int nb_or;
    bool deplacable;
};
\end{lst-c++}

\noindent
\begin{tabular}[t]{@{\extracolsep{0pt}}>{\bfseries}lp{10cm}}
Description~: & Représente un bateau \\
Champs~: &
\small
\begin{tabular}[t]{@{\extracolsep{0pt}}lp{7cm}}
    
        \textsl{id}~: & Identifiant unique du bateau \\
    
        \textsl{pos}~: & Position \\
    
        \textsl{joueur}~: & Joueur \\
    
        \textsl{btype}~: & Type \\
    
        \textsl{nb\_or}~: & Or contenu dans le bateau \\
    
        \textsl{deplacable}~: & Le bateau n'a pas encore été déplacé ce tour-ci \\
    
\end{tabular} \\
\end{tabular}




\begin{minipage}{\linewidth}
\functitle{info\_terrain}

\begin{lst-c++}
terrain info_terrain(position pos)
\end{lst-c++}

\noindent
\begin{tabular}[t]{@{\extracolsep{0pt}}>{\bfseries}lp{10cm}}
Description~: & Retourne la nature du terrain désigné par ``pos``. \\


Parametres~: &
\begin{tabular}[t]{@{\extracolsep{0pt}}ll}
    
    
      
        \textsl{pos}~: & Position \\
      
    
  \end{tabular} \\






\end{tabular} \\[0.3cm]
\end{minipage}


\begin{minipage}{\linewidth}
\functitle{info\_ile\_joueur}

\begin{lst-c++}
int info_ile_joueur(position pos)
\end{lst-c++}

\noindent
\begin{tabular}[t]{@{\extracolsep{0pt}}>{\bfseries}lp{10cm}}
Description~: & Retourne le joueur qui possède l'île à l'emplacement ``pos``. Retourne -1 si l'île est libre ou si la position indiquée n'est pas une île \\


Parametres~: &
\begin{tabular}[t]{@{\extracolsep{0pt}}ll}
    
    
      
        \textsl{pos}~: & Position \\
      
    
  \end{tabular} \\






\end{tabular} \\[0.3cm]
\end{minipage}


\begin{minipage}{\linewidth}
\functitle{info\_ile\_or}

\begin{lst-c++}
int info_ile_or(position pos)
\end{lst-c++}

\noindent
\begin{tabular}[t]{@{\extracolsep{0pt}}>{\bfseries}lp{10cm}}
Description~: & Retourne l'or contenu sur l'île à l'emplacement ``pos``. Retourne -1 si la case spécifiée n'est pas une île. \\


Parametres~: &
\begin{tabular}[t]{@{\extracolsep{0pt}}ll}
    
    
      
        \textsl{pos}~: & Position \\
      
    
  \end{tabular} \\






\end{tabular} \\[0.3cm]
\end{minipage}


\begin{minipage}{\linewidth}
\functitle{info\_bateau}

\begin{lst-c++}
bateau info_bateau(int id)
\end{lst-c++}

\noindent
\begin{tabular}[t]{@{\extracolsep{0pt}}>{\bfseries}lp{10cm}}
Description~: & Retourne le bateau ayant pour identifiant ``id`` \\


Parametres~: &
\begin{tabular}[t]{@{\extracolsep{0pt}}ll}
    
    
      
        \textsl{id}~: & Identifiant \\
      
    
  \end{tabular} \\






\end{tabular} \\[0.3cm]
\end{minipage}


\begin{minipage}{\linewidth}
\functitle{bateau\_existe}

\begin{lst-c++}
bool bateau_existe(int id)
\end{lst-c++}

\noindent
\begin{tabular}[t]{@{\extracolsep{0pt}}>{\bfseries}lp{10cm}}
Description~: & Retourne vrai si le bateau ayant pour identifiant ``id`` existe et est encore à flots \\


Parametres~: &
\begin{tabular}[t]{@{\extracolsep{0pt}}ll}
    
    
      
        \textsl{id}~: & Identifiant \\
      
    
  \end{tabular} \\






\end{tabular} \\[0.3cm]
\end{minipage}


\begin{minipage}{\linewidth}
\functitle{liste\_bateaux\_position}

\begin{lst-c++}
bateau array liste_bateaux_position(position pos)
\end{lst-c++}

\noindent
\begin{tabular}[t]{@{\extracolsep{0pt}}>{\bfseries}lp{10cm}}
Description~: & Retourne la liste de bateaux à la position ``pos`` \\


Parametres~: &
\begin{tabular}[t]{@{\extracolsep{0pt}}ll}
    
    
      
        \textsl{pos}~: & Position \\
      
    
  \end{tabular} \\






\end{tabular} \\[0.3cm]
\end{minipage}


\begin{minipage}{\linewidth}
\functitle{liste\_id\_bateaux\_position}

\begin{lst-c++}
int array liste_id_bateaux_position(position pos)
\end{lst-c++}

\noindent
\begin{tabular}[t]{@{\extracolsep{0pt}}>{\bfseries}lp{10cm}}
Description~: & Retourne la liste des ID des bateaux à la position ``pos`` \\


Parametres~: &
\begin{tabular}[t]{@{\extracolsep{0pt}}ll}
    
    
      
        \textsl{pos}~: & Position \\
      
    
  \end{tabular} \\






\end{tabular} \\[0.3cm]
\end{minipage}


\begin{minipage}{\linewidth}
\functitle{liste\_iles}

\begin{lst-c++}
position array liste_iles()
\end{lst-c++}

\noindent
\begin{tabular}[t]{@{\extracolsep{0pt}}>{\bfseries}lp{10cm}}
Description~: & Retourne la liste des positions des îles de la carte \\







\end{tabular} \\[0.3cm]
\end{minipage}


\begin{minipage}{\linewidth}
\functitle{mes\_iles}

\begin{lst-c++}
position array mes_iles()
\end{lst-c++}

\noindent
\begin{tabular}[t]{@{\extracolsep{0pt}}>{\bfseries}lp{10cm}}
Description~: & Retourne la liste des positions des îles qui vous appartiennent \\







\end{tabular} \\[0.3cm]
\end{minipage}


\begin{minipage}{\linewidth}
\functitle{distance}

\begin{lst-c++}
int distance(position depart, position arrivee)
\end{lst-c++}

\noindent
\begin{tabular}[t]{@{\extracolsep{0pt}}>{\bfseries}lp{10cm}}
Description~: & Retourne la distance entre deux positions \\


Parametres~: &
\begin{tabular}[t]{@{\extracolsep{0pt}}ll}
    
    
      
        \textsl{depart}~: & Départ \\
      
    
      
        \textsl{arrivee}~: & Arrivée \\
      
    
  \end{tabular} \\






\end{tabular} \\[0.3cm]
\end{minipage}


\begin{minipage}{\linewidth}
\functitle{construire}

\begin{lst-c++}
erreur construire(bateau_type btype, position pos)
\end{lst-c++}

\noindent
\begin{tabular}[t]{@{\extracolsep{0pt}}>{\bfseries}lp{10cm}}
Description~: & Construire un bateau de type ``btype`` sur l'île à la position ``pos`` \\


Parametres~: &
\begin{tabular}[t]{@{\extracolsep{0pt}}ll}
    
    
      
        \textsl{btype}~: & Type de bateau à construire \\
      
    
      
        \textsl{pos}~: & Position \\
      
    
  \end{tabular} \\






\end{tabular} \\[0.3cm]
\end{minipage}


\begin{minipage}{\linewidth}
\functitle{deplacer}

\begin{lst-c++}
erreur deplacer(int id, position pos)
\end{lst-c++}

\noindent
\begin{tabular}[t]{@{\extracolsep{0pt}}>{\bfseries}lp{10cm}}
Description~: & Déplace le bateau représenté par l'identifiant ``id`` jusqu'à la position `pos`` (si elle est dans la portée du bateau) \\


Parametres~: &
\begin{tabular}[t]{@{\extracolsep{0pt}}ll}
    
    
      
        \textsl{id}~: & Identifiant du bateau \\
      
    
      
        \textsl{pos}~: & Position sur laquelle déplacer le bateau \\
      
    
  \end{tabular} \\






\end{tabular} \\[0.3cm]
\end{minipage}


\begin{minipage}{\linewidth}
\functitle{coloniser}

\begin{lst-c++}
erreur coloniser(position pos)
\end{lst-c++}

\noindent
\begin{tabular}[t]{@{\extracolsep{0pt}}>{\bfseries}lp{10cm}}
Description~: & Colonise l'île à la position ``pos`` \\


Parametres~: &
\begin{tabular}[t]{@{\extracolsep{0pt}}ll}
    
    
      
        \textsl{pos}~: & Position de l'île à coloniser \\
      
    
  \end{tabular} \\






\end{tabular} \\[0.3cm]
\end{minipage}


\begin{minipage}{\linewidth}
\functitle{charger}

\begin{lst-c++}
erreur charger(int id, int nb_or)
\end{lst-c++}

\noindent
\begin{tabular}[t]{@{\extracolsep{0pt}}>{\bfseries}lp{10cm}}
Description~: & Charge la caravelle identifiée par ``id`` de ``nb\_or`` d'or. \\


Parametres~: &
\begin{tabular}[t]{@{\extracolsep{0pt}}ll}
    
    
      
        \textsl{id}~: & Identifiant de la caravelle \\
      
    
      
        \textsl{nb\_or}~: & Quantité d'or à charger \\
      
    
  \end{tabular} \\






\end{tabular} \\[0.3cm]
\end{minipage}


\begin{minipage}{\linewidth}
\functitle{decharger}

\begin{lst-c++}
erreur decharger(int id, int nb_or)
\end{lst-c++}

\noindent
\begin{tabular}[t]{@{\extracolsep{0pt}}>{\bfseries}lp{10cm}}
Description~: & Décharge la caravelle identifiée par ``id`` de ``nb\_or`` d'or. \\


Parametres~: &
\begin{tabular}[t]{@{\extracolsep{0pt}}ll}
    
    
      
        \textsl{id}~: & Identifiant de la caravelle \\
      
    
      
        \textsl{nb\_or}~: & Quantité d'or à décharger \\
      
    
  \end{tabular} \\






\end{tabular} \\[0.3cm]
\end{minipage}


\begin{minipage}{\linewidth}
\functitle{mon\_joueur}

\begin{lst-c++}
int mon_joueur()
\end{lst-c++}

\noindent
\begin{tabular}[t]{@{\extracolsep{0pt}}>{\bfseries}lp{10cm}}
Description~: & Retourne le numéro de votre joueur \\







\end{tabular} \\[0.3cm]
\end{minipage}


\begin{minipage}{\linewidth}
\functitle{adversaire}

\begin{lst-c++}
int adversaire()
\end{lst-c++}

\noindent
\begin{tabular}[t]{@{\extracolsep{0pt}}>{\bfseries}lp{10cm}}
Description~: & Retourne le numéro de votre adversaire \\







\end{tabular} \\[0.3cm]
\end{minipage}


\begin{minipage}{\linewidth}
\functitle{score}

\begin{lst-c++}
int score(int id_joueur)
\end{lst-c++}

\noindent
\begin{tabular}[t]{@{\extracolsep{0pt}}>{\bfseries}lp{10cm}}
Description~: & Retourne les scores du joueur désigné par l'identifiant ``id`` \\


Parametres~: &
\begin{tabular}[t]{@{\extracolsep{0pt}}ll}
    
    
      
        \textsl{id\_joueur}~: & Identifiant du joueur \\
      
    
  \end{tabular} \\






\end{tabular} \\[0.3cm]
\end{minipage}


\begin{minipage}{\linewidth}
\functitle{tour\_actuel}

\begin{lst-c++}
int tour_actuel()
\end{lst-c++}

\noindent
\begin{tabular}[t]{@{\extracolsep{0pt}}>{\bfseries}lp{10cm}}
Description~: & Retourne le numéro du tour actuel \\







\end{tabular} \\[0.3cm]
\end{minipage}


